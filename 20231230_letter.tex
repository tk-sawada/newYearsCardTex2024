\documentclass{ltjsarticle}
%\documentclass{jlreq}
%\documentclass[fontsize=10.5bp]{jlreq}
\usepackage[margin=10mm, paperwidth=100mm, paperheight=148mm]{geometry}
%\usepackage[haranoaji,deluxe]{luatexja-preset}
%\usepackage{luatexja-otf}
%\usepackage[math]{iwona}
%\usepackage{tikz}
%\usetikzlibrary{shapes.geometric}
\usepackage[no-math,deluxe,expert,haranoaji]{luatexja-preset}
\usepackage[math]{iwona}
\usepackage{xcolor}
\setmainfont{HaranoAjiGothic}
\setsansfont{iwona}

\pagestyle{empty}
\setlength\parindent{0pt}
\begin{document}


\begin{center}
    % 謹賀新年
    {\fontsize{50pt}{60pt} \gtfamily{謹賀{\color{red}{新}}年}}
    % 空白調整
    \vspace{0.3cm}
    % 2024
    {\fontsize{42pt}{45pt} $2024$}
    % 「2016」と最後の文章の間の水平方向の空白
    \vspace{0.6cm}
    {\fontsize{200pt}{240pt} \gtfamily{辰}}
    \vspace{0.6cm}
  \end{center}
めちゃくちゃテンプレートぽい!って思いましたね?いちから作りました。


\end{document}
